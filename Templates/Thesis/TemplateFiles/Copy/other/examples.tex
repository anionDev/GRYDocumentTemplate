% !TeX root = ../document.tex

\chapter{Examples}

\section{Special character}\label{sec_specialcharacter}

example: \textbackslash \textasciitilde \\
a<b>c|d/'`¿ÇäöüÄÖÜßé@!23\\
this is some \enquote{quoted text}.


\section{References}

This sentence also has a reference to a bibliography-item (book).\cite{bib_zebk}\\
This sentence also has a reference to a bibliography-item (online).\cite{bib_amtoc}\\
This sentence has a simple footnote.\footnote{some footnote-text}\\
This is the usage of a glossary-entry: \gls{glossaryentry1}\\
This is the usage of another glossary-entry: \gls{glossaryentry2}\\
This is the first usage of an abbreviation: \gls{phd}\\
This is the first usage of another abbreviation: \gls{msc}\\
This is the second usage of an abbreviation: \gls{phd}\\
This is the second usage of another abbreviation: \gls{msc}\\
This is a reference to figure 1: \ref{figure_example1}\\
This is a reference to table 2: \ref{tables_table2}\\
This is a reference to listing 1: \ref{listings_listing1}\\
This is a reference to a section: \ref{sec_specialcharacter}\\
(All usages of the \enquote{ref}-command require a \enquote{target}-annotation on the target-element.)

\section{Lists}

itemize (unordered):
\begin{itemize}
    \item This is an entry
    \item This is another entry
\end{itemize}
enumerate (ordered):
\begin{enumerate}
    \item This is an entry
    \item This is another entry
\end{enumerate}

\section{Figures}

A figure:
\begin{figure}[H]
    \centering
    \includegraphics{figures/examples/example1.png}
    \caption{Figure 1 caption}\label{figure_example1}
\end{figure}

another figure:
\begin{figure}[H]
    \centering
    \includegraphics{figures/examples/example2.png}
    \caption{Figure 2 caption}\label{figure_example2}
\end{figure}

\section{Tables}

A table (without border):
\begin{table}[H]
    \centering
    \begin{tabular}{ l c r }
        1 & 2 & 3 \\
        4 & 5 & 6 \\
        7 & 8 & 9 \\
    \end{tabular}
    \caption{Table 1 caption}\label{tables_table1}
\end{table}

Another table (with border):
\begin{table}[H]
    \centering
    \begin{tabular}{ | l | c | r | }\hline
        1 & 2 & 3 \\\hline
        4 & 5 & 6 \\\hline
        7 & 8 & 9 \\\hline
    \end{tabular}
    \caption{Table 2 caption}\label{tables_table2}
\end{table}

\section{Listings}

A listing:
\begin{code}
    \captionof{listing}{A simple docker-compose.yml-file}\label{listings_listing1}
    \inputminted[
        frame=single,
        obeytabs=true,
        tabsize=4,
        linenos,
        breaklines,
        numbersep=6pt,
        breakanywhere
    ]{yaml}{other/listings/examples/SimpleDockerCompose.yml}
\end{code}

Another listing:
\begin{code}
    \captionof{listing}{A simple Python-program}\label{listings_listing2}
    \inputminted[
        frame=single,
        obeytabs=true,
        tabsize=4,
        linenos,
        breaklines,
        numbersep=6pt,
        breakanywhere
    ]{bash}{other/listings/examples/SimplePythonScript.py}
\end{code}
